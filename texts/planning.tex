%++++++++++++++++++++++++++++++++++++++++
% Don't modify this section unless you know what you're doing!
\documentclass[letterpaper,12pt]{article}
\usepackage{tabularx} % extra features for tabular environment
\usepackage{amsmath}  % improve math presentation
\usepackage{graphicx} % takes care of graphic including machinery
\usepackage[margin=1in,letterpaper]{geometry} % decreases margins
\usepackage{cite} % takes care of citations
\usepackage[final]{hyperref} % adds hyper links inside the generated pdf file
\hypersetup{
	colorlinks=true,       % false: boxed links; true: colored links
	linkcolor=blue,        % color of internal links
	citecolor=blue,        % color of links to bibliography
	filecolor=magenta,     % color of file links
	urlcolor=blue         
}
%++++++++++++++++++++++++++++++++++++++++

% Packages:
\usepackage{amsmath}
\usepackage{amsfonts}    % para mathbb
\usepackage{mathrsfs}     % para usar mathscr. Hay que incluir ...?
\usepackage{mathabx}     % para el simbolo de transformada de Fourier inversa.
\usepackage{caption}       % para etiquetas en figuras y tablas
\usepackage{pifont}          % Checkmark and x mark
\usepackage[T1]{fontenc} % To add several authors
\usepackage{enumitem}   % enumerate issues
\usepackage{appendix}
\usepackage{cellspace}   % In particular, allows to properly use $\dfrac{}{}$ expressions inside tables
\setlength\cellspacetoplimit{3pt}
\setlength\cellspacebottomlimit{3pt}
\usepackage{xcolor} % to write in different colours
\newcommand{\colred}[1]{\color{red} #1 \color{black}}
\newcommand{\colblue}[1]{\color{blue} #1 \color{black}}
\usepackage{lineno}

%%%% several %%%%:
\newcommand{\dsty}[1]{\displaystyle{#1}}
\newcommand{\bs}[1]{\boldsymbol{#1}}
\newcommand{\sph}[1]{\langle {#1} \rangle}

%%%% \beguin y \end %%%%
% 1) equation
\newcommand{\beq}[0]{ \begin{equation} }
\newcommand{\beql}[2]{ \begin{equation} \label{#1} }
\newcommand{\eeq}[0]{ \end{equation} }
% 2) align
\newcommand{\bal}[0]{ \begin{align} }
\newcommand{\eal}[0]{ \end{align} }
% 3) align*
\newcommand{\baln}[0]{ \begin{align*} }
\newcommand{\ealn}[0]{ \end{align*} }
% 4) enumerate
\newcommand{\benum}[0]{ \begin{enumerate} }
\newcommand{\eenum}[0]{ \end{enumerate} }
% 5) itemize
\newcommand{\bitem}[0]{ \begin{itemize} }
\newcommand{\eitem}[0]{ \end{itemize} }
%%%% Commands %%%
% 1) Partial derivatives:
\newcommand{\p}[0]{ \partial}
\newcommand{\dpar}[2]{ \dfrac{\partial {#1}}{\partial {#2}} }
\newcommand{\ddpar}[2]{ \dfrac{\partial^2 {#1}}{\partial {#2}^2} }
\newcommand{\dppar}[3]{ \dfrac{\partial^{#1}{#2}}{\partial {#3}^{#1}} }
% 2) Symbols
\newcommand{\cmark}{\ding{51}}%
\newcommand{\xmark}{\ding{55}}%
% 3) Integrals:
\newcommand{\dint}{\displaystyle\int}
\newcommand{\dintR}{\displaystyle\int_{\mathbb{R}}}
\newcommand{\dintRd}{\displaystyle\int_{\mathbb{R}^d}}
%%%% Theoremes %%%
\newtheorem{theorem}{Theorem}[section]
\newtheorem{corollary}{Corollary}[theorem]
\newtheorem{lemma}[theorem]{Lemma}
\newtheorem{defi}{Definition}[section]
%%%% Color box: 
\usepackage{xcolor}
\definecolor{softgray}{gray}{0.85}
\usepackage{tcolorbox}
\tcbset{colframe=white}
%\newtcolorbox{mycolorbox}[1][]{capture=hbox,#1,colback=softgray,sharpish corners}
\newtcolorbox{mycolorbox}[1][]{colback=softgray,sharpish corners}
%%%% Fourier Transform
\usepackage{scalerel,stackengine}
\stackMath
\newcommand\reallywidehat[1]{%
\savestack{\tmpbox}{\stretchto{%
  \scaleto{%
    \scalerel*[\widthof{\ensuremath{#1}}]{\kern.1pt\mathchar"0362\kern.1pt}%
    {\rule{0ex}{\textheight}}%WIDTH-LIMITED CIRCUMFLEX
  }{\textheight}% 
}{2.4ex}}%
\stackon[-6.9pt]{#1}{\tmpbox}%
}
\parskip 1ex
%%%%%%%%%%%%%%%%%%%%%%%%%%%%%%%%%%%%%%%%%%%%%%%%%%%%%%%%%%%%%%%%%%%%%%%%%%%%%
\begin{document}
\title{\Huge{\textbf{Plan de trabajo XRagua}} \\ Plan de trabajo}
\author{Josep Angel Vicente, Pablo Eleazar Merino Alonso}
\date{\today}
\maketitle
%\begin{abstract}
%  --- Abstract here ---
%\end{abstract}
%%%%%%%%%%%%%%%%%%%%%%%%%%%%%%%%%%%%%%%%%%%%%%%%%%%%%%%%%%%%%%%%%%%%%%%%%%%%%
\section{Organizaci\'on del documento}
\label{section-intro}
Plan de trabajo para los pr\'oximos 6 meses. La secci\'on \ref{section-plan-tema} presenta el planning por tem\'atica. La secci\'on \ref{section-plan-temp} incluye un resumen cronol\'ogico del planning.
La primera secci\'on est\'a dedicada a un planteamiento general de los objetivos.
Para cada Actuaci\'on del planning se detallan: 
\begin{itemize}
\item Descripcci\'on y relaci\'on con el trabajo realizado hasta ahora. 
\item Estado actual.
\item Importancia y \colblue{objetivo temporal}. 
\end{itemize}
%%%%%%%%%%%%%%%%%%%%%%%%%%%%%%%%%%%%%%%%%%%%%%%%%%%%%%%%%%%%%%%%%%%%%%%%%%%%%
\section{Estrategia general}
\label{section-estrategia}
El objetivo a plazo de 6 meses es tener un trabajo para presentarlo en el XIFU CM sobre separaci\'on de fuentes, tiempos de exposici\'on, etc. En esencia, estrategias de observaci\'on par XIFU. La ventaja que tenemos frente a otros grupos de trabajo es el uso de xifusim o sixte (que mucha gente no sabe utilizar).
%%%%%%%%%%%%%%%%%%%%%%%%%%%%%%%%%%%%%%%%%%%%%%%%%%%%%%%%%%%%%%%%%%%%%%%%%%%%%
\section{Planning por tem\'atica}
\label{section-plan-tema}
\begin{enumerate}
%%%%%%%%%%%%%%%%%%%%%%%%%%%%%%%%%%%%%%%%%%%%%%%%%%
\item \textbf{Estrategias de observaci\'on con XIFU} Prioridad alta. 
\benum
%%%%%%%%%%%%%%%%%%
\item \textbf{Separaci\'on de fuentes}. Lo que hay hecho: hemos simulado cinco fuentes con un solo simput, y extra\'ido los espectros pixel a pixel. Las subtareas son:
\benum
\item Visualizar los espectros de las 5 fuentes por separado y juntas. Generar una imagen de las cinco fuentes. DONE.
\item 
\label{item-obser-sep-2sources}
Ejercicio con dos fuentes: crear un script que simule la observaci\'on; genere la imagen; genere los espectros por separado y juntos. Comprobar que funciona bien.
\item
\label{item-observ-array-realcases} 
Crear un script que genere un array de casos en los que var\'ie: ratio de flujos, separaci\'on, espectro y tiempos de exposici\'on. Para ello el script contendr\'a un bucle m\'ultiple y llamar\'a al script del punto \ref{item-obser-sep-2sources} llame al anterior pero con distintos par\'ametros.
\item
\label{item-observ-science-interest}
Buscar casos de inter\'es cient\'ifico. 
\item Ajustar el array de casos con parámetros de estadísticos y calcular los intervalos de confianza
\eenum
%%%%%%%%%%%%%%%%%%
\item
\label{time-triplets}
\textbf{Tiempos de exposici\'on}. Estudiar para los tripletes del SiXIII, MgXI, NeIX, la relaci\'on entre los tiempos de exposici\'on y la certidumbre de las intensidades de las l\'ineas $r$, $f$, $i$ y la medida de los par\'ametros $R$ y $G$. Esta tarea se estructura en:
\benum
\item
\label{simput-isis}
\textbf{Generaci\'on de Simput con ISIS}. Ese script hay que intentar adaptarlo para poder usarlo con distintos par\'ametros. Pensar sobre ello. 
\item
\label{triplet-spectre} 
\textbf{Localizar los espectros} en formatos .dat de los tres tripletes.
\item
\label{watanabe}
Buscar en el art\'iculo de \textbf{Watanabe} los flujos de las l\'ineas de los tripletes para Vela X-I y comprobar que los flujos est\'an bien en los .dat.
\item
\label{triplet-power}
Investigar c\'omo generar un solo espectro donde est\'en los tres  con un powerlaw (ajustar el powerlaw para ver bien los tripletes). 
\item
\label{expo-array}
Crear un script que genere un array de casos para distintos tiempos de exposici\'on de un espectro (o tres) que contenga los tres tripletes.
\item
\label{fit-isis}
\textbf{Ajuste de los datos}. Localizar los scripts de ISIS que hacen un pseudoajuste autom\'atico.
\item
\label{fit-ci}
Calcular los \textbf{intervalos de confianza} para los flujos de las l\'ineas. 
\eenum
%%%%%%%%%%%%%%%%%%
\item \textbf{Simulaci\'on de un clusters}
\benum
\item Encontrar clusters con fuentes de rayos X de inter\'es. 
\item
\label{cluster-simput}
\textbf{Aprender a generar con ISIS un simput con muchas fuentes}
\item Pensar c\'omo podemos explotar \'este caso. Se pueden hacer muchas cosas aqu\'i. 
\eenum
%%%%%%%%%%%%%%%%%%
\item 
\label{paper-xcm}
\textbf{Generaci\'on de los documentos}. Escribir todo en un paper. Preparar la presentaci\'on para XCM pr\'oximo. Pensar en qu\'e revistas se puede publicar esto. Pensar si se puede ir con ello al SPIE. 
\eenum
%%%%%%%%%%%%%%%%%%%%%%%%%%%%%%%%%%%%%%%%%%%%%%%%%%
\item \textbf{Ajuste de datos}. 
\benum
\item Familiarizarse con el uso de ISIS. Hacer el tutorial de Bamberg. (PEMA, JAVO) Prioridad alta.
\item Organizar un seminario sobre ajuste de datos. Difundir entre los alumnos. (PEMA) Prioridad alta. 
\item 
\label{python-fit}
Buscar otras herramientas de fitting distintas de fitting (ver python).
\item
\label{bayes}
Pensar si es interesante estudiar \textbf{estad\'istica bayesiana}. 
\item Estudiar como ajustar datos usando \textbf{redes neuronales}. 
\item Pensar si es interesante plantearse hacer estudios (y papers) quie consistan en ajuste de datos de otras tem\'aticas. (No queremos gastar mucha energ\'ia en eso, s\'olo si es asequible en poco tiempo).
\item Pensar si organizar otra asignatura Rafael Altamira sobre ajuste de datos. 
\eenum
%%%%%%%%%%%%%%%%%%%%%%%%%%%%%%%%%%%%%%%%%%%%%%%%%%
\item \textbf{Xifusim-Sixte}. Prioridad alta. Esperamos a ver qu\'e dice Joern. 
%%%%%%%%%%%%%%%%%%%%%%%%%%%%%%%%%%%%%%%%%%%%%%%%%%
\item
\label{difusion}
\textbf{Difusi\'on}. Es muy importante dar visibilidad al grupo en todo lo que hacemos. Entre la comunidad cient\'ifica y entre los estudiantes (de la UA, de la UMH... ). Nos interesa: twitter, facebook, ResearchGate, Academia.edu, Radio, Diario Informaci\'on u otros peri\'odicos. Organizar seminarios a los alumnos, etc. 
\benum
\item Pasar a Josep las cuentas de Facebook y Twiter. (Prioridad alta)
\item Hacer unos posts anunciando las actualizaciones en la web. (Prioridad alta)
\item Estudiar como organizarlo en ResearchGate y Academia. 
\item Ver si hacer Radio / televisi\'on / Prensa. 
\item Pensar en organizar seminarios o charlas para los alumnos. 
\item Pensar en como los alumnos pueden hacer tfgs con el grupo. Pensar temas y proponerlos. PEMA
\item Pensar sobre asignatura verano Rafael Altamira. 
\item Pensar si organizar otra asignatura Rafael Altamira sobre ajuste de datos. 
\item
\label{selenium}
Automizar Redes sociales. Libreria python: selenium. Investigalo. (Tenerlo en un mes)
\item JJ ha vuelto a actualizar la web del grupo. 
\eenum
%%%%%%%%%%%%%%%%%%%%%%%%%%%%%%%%%%%%%%%%%%%%%%%%%%
\item \textbf{Organizar todo con Git}. 
\end{enumerate}
%%%%%%%%%%%%%%%%%%%%%%%%%%%%%%%%%%%%%%%%%%%%%%%%%%%%%%%%%%%%%%%%%%%%%%%%%%%%%
\section{Planning temporal}
\label{section-plan-temp}
\subsection*{Semana 4-8 Mayo}
Script de dos fuentes y un script de arrays de casos. \ref{item-obser-sep-2sources} , \ref{item-observ-array-realcases}
\subsection*{Semana 11-15 Mayo}
Continuación de \ref{item-observ-array-realcases} y estudiar casos de interés científico en \ref{item-observ-science-interest}
\\Automatizar RRSS \ref{selenium}
\subsection*{Semana 18-22 Mayo}
Estudiar tripletes y Simputs con ISIS. \ref{time-triplets} , \ref{simput-isis}
\subsection*{Semana 25-29 Mayo}
Localizar los espectros en \textit{.dat} y estudiar el \textit{Watanabe}. \ref{triplet-spectre} , \ref{watanabe}
\subsection*{1ª Quincena Junio}
Generar espectro con \textit{Powerlaw+tripletes}. Array de casos con tripletes \ref{expo-array} , \ref{triplet-power}
\subsection*{2º Quincena Junio}
Ajuste de datos e intervalos de confianza de par\'ametros. \ref{fit-isis} , \ref{fit-ci}
\subsection*{1ª Quincena Julio}
Simulaci\'on de clusters. \ref{cluster-simput}
\subsection*{2º Quincena de Agosto}
Realizar un paper sobre los resultados. \ref{paper-xcm}
\subsection*{Septiembre}
Preparar la presentaci\'on XCM.\ref{paper-xcm}
\\Difusi\'on al alumnado \ref{difusion}
\\Aprender estad\'istica bayesiana \ref{bayes}
\subsection*{Octubre}
Ajuste de datos, aprender librerias de python \ref{python-fit}
%%%%%%%%%%%%%%%%%%%%%%%%%%%%%%%%%%%%%%%%%%%%%%%%%%%%%%%%%%%%%%%%%%%%%%%%%%%%%
%\bibliographystyle{IEEEtran.bst}
%\bibliography{./bib} % local
\end{document}