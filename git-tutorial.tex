%++++++++++++++++++++++++++++++++++++++++
% Don't modify this section unless you know what you're doing!
\documentclass[letterpaper,12pt]{article}
\usepackage{tabularx} % extra features for tabular environment
\usepackage{amsmath}  % improve math presentation
\usepackage{graphicx} % takes care of graphic including machinery
\usepackage[margin=1in,letterpaper]{geometry} % decreases margins
\usepackage{cite} % takes care of citations
\usepackage[final]{hyperref} % adds hyper links inside the generated pdf file
\hypersetup{
	colorlinks=true,       % false: boxed links; true: colored links
	linkcolor=blue,        % color of internal links
	citecolor=blue,        % color of links to bibliography
	filecolor=magenta,     % color of file links
	urlcolor=blue
}
%++++++++++++++++++++++++++++++++++++++++

% Packages:
\usepackage{amsmath}
\usepackage{amsfonts}    % para mathbb
\usepackage{mathrsfs}     % para usar mathscr. Hay que incluir ...?
\usepackage{mathabx}     % para el simbolo de transformada de Fourier inversa.
\usepackage{caption}       % para etiquetas en figuras y tablas
\usepackage{pifont}          % Checkmark and x mark
\usepackage[T1]{fontenc} % To add several authors
\usepackage{enumitem}   % enumerate issues
\usepackage{appendix}
\usepackage{cellspace}   % In particular, allows to properly use $\dfrac{}{}$ expressions inside tables
\setlength\cellspacetoplimit{3pt}
\setlength\cellspacebottomlimit{3pt}
\usepackage{xcolor} % to write in different colours
\newcommand{\colred}[1]{\color{red} #1 \color{black}}
\newcommand{\colblue}[1]{\color{blue} #1 \color{black}}
\usepackage{lineno}

%%%% several %%%%:
\newcommand{\dsty}[1]{\displaystyle{#1}}
\newcommand{\bs}[1]{\boldsymbol{#1}}
\newcommand{\sph}[1]{\langle {#1} \rangle}

%%%% \beguin y \end %%%%
% 1) equation
\newcommand{\beq}[0]{ \begin{equation} }
\newcommand{\beql}[2]{ \begin{equation} \label{#1} }
\newcommand{\eeq}[0]{ \end{equation} }
% 2) align
\newcommand{\bal}[0]{ \begin{align} }
\newcommand{\eal}[0]{ \end{align} }
% 3) align*
\newcommand{\baln}[0]{ \begin{align*} }
\newcommand{\ealn}[0]{ \end{align*} }
% 4) enumerate
\newcommand{\benum}[0]{ \begin{enumerate} }
\newcommand{\eenum}[0]{ \end{enumerate} }
% 5) itemize
\newcommand{\bitem}[0]{ \begin{itemize} }
\newcommand{\eitem}[0]{ \end{itemize} }
%%%% Commands %%%
% 1) Partial derivatives:
\newcommand{\p}[0]{ \partial}
\newcommand{\dpar}[2]{ \dfrac{\partial {#1}}{\partial {#2}} }
\newcommand{\ddpar}[2]{ \dfrac{\partial^2 {#1}}{\partial {#2}^2} }
\newcommand{\dppar}[3]{ \dfrac{\partial^{#1}{#2}}{\partial {#3}^{#1}} }
% 2) Symbols
\newcommand{\cmark}{\ding{51}}%
\newcommand{\xmark}{\ding{55}}%
% 3) Integrals:
\newcommand{\dint}{\displaystyle\int}
\newcommand{\dintR}{\displaystyle\int_{\mathbb{R}}}
\newcommand{\dintRd}{\displaystyle\int_{\mathbb{R}^d}}
%%%% Theoremes %%%
\newtheorem{theorem}{Theorem}[section]
\newtheorem{corollary}{Corollary}[theorem]
\newtheorem{lemma}[theorem]{Lemma}
\newtheorem{defi}{Definition}[section]
%%%% Color box:
\usepackage{xcolor}
\definecolor{softgray}{gray}{0.85}
\usepackage{tcolorbox}
\tcbset{colframe=white}
%\newtcolorbox{mycolorbox}[1][]{capture=hbox,#1,colback=softgray,sharpish corners}
\newtcolorbox{mycolorbox}[1][]{colback=softgray,sharpish corners}
%%%% Fourier Transform
\usepackage{scalerel,stackengine}
\stackMath
\newcommand\reallywidehat[1]{%
\savestack{\tmpbox}{\stretchto{%
  \scaleto{%
    \scalerel*[\widthof{\ensuremath{#1}}]{\kern.1pt\mathchar"0362\kern.1pt}%
    {\rule{0ex}{\textheight}}%WIDTH-LIMITED CIRCUMFLEX
  }{\textheight}%
}{2.4ex}}%
\stackon[-6.9pt]{#1}{\tmpbox}%
}
\parskip 1ex
%%%%%%%%%%%%%%%%%%%%%%%%%%%%%%%%%%%%%%%%%%%%%%%%%%%%%%%%%%%%%%%%%%%%%%%%%%%%%
\begin{document}
\title{\Huge{\textbf{Tutorial de Git}} \\ Herramienta de Trabajo}
\author{Josep Angel Vicente, Pablo Eleazar Merino Alonso}
\date{\today}
\maketitle
%\begin{abstract}
%  --- Abstract here ---
%\end{abstract}
%%%%%%%%%%%%%%%%%%%%%%%%%%%%%%%%%%%%%%%%%%%%%%%%%%%%%%%%%%%%%%%%%%%%%%%%%%%%%
\section{Contexto}
\textbf{GIT} es un sistema de control de versiones de archivos de programación, dedicado al seguimiento de cambios en dichos programas
\begin{itemize}
\item  Coordina el trabajo entre diferentes desarrolladores.
\item Identificador de qué cambios se han hecho en el programa y la fecha de modificación
\item Repositorios locales y remotos
\item Puntos de guardado y modificación tanto en repositorios local or remote
\end{itemize}

\section{Inicio}
\textbf{GIT} instala en nuestro sistema UNIX, un terminal bash, con mejores funcionalidades. En \textbf{GIT} existen 3 tipos de Estados
\begin{itemize}
\item \textbf{Working Directory}: Estado de trabajo de todos los archivos
\item \textbf{Staging Area}: Estado de agregación para añadir al guardado en \textit{GIT}
\item \textbf{Repository}: Estado de cambios definitivos para el repositorio remoto.
\end{itemize}
\begin{figure}[!b]
	\centering
  \includegraphics[scale=0.8]{basic.jpg}
  \caption{3 Estados de \textit{GIT}}
  \label{fig:basic}
\end{figure}
\subsection{Comandos Básicos}
Aquí aparecerán los más importantes
\begin{itemize}
\item \textbf{\textit{git init}} Proyecto Nuevo
\item \textbf{\textit{git add <file>}} Enviar los archivos que deseas del directorio de trabajo al \textit{staging area}
\item \textbf{\textit{git status}} Visualización del estado de tus archivos
\item \textbf{\textit{git commit}} Envío de archivos desde el \textit{staging area} hacia el repositorio, es un modo de manifestar un snapshot de la versión de tu código de trabajo
\item \textbf{\textit{git push}} Envío de archivos al repositorio remoto, donde ya tienes una versión definitiva que quieres compartir con los demás desarrolladores que tiene permiso para modificar el código
\item \textbf{\textit{git pull}} Comando que te dice que cambios han realizado los demás desarrolladores
\item \textbf{\textit{git clone}}Copia desde el servidor central hasta el repositorio local

\end{itemize}

\section{Descargando el Remote}
Partiendo de un directorio, en la consola
\begin{verbatim}
user@user $ git clone <url path>
\end{verbatim}
Podemos crear varios archivos y directorios a conveniencia, una vez hecho para subirlos al remoto, debemos usar...
\begin{verbatim}
$ git status  %Para ver que los archivos ahora estan para subir
$ git add <file> %añade el archivo al staging area
\end{verbatim}
Una herramienta de visualización de cambios de un archivo que se ha modificado es
\begin{verbatim}
$ git diff
\end{verbatim}
Esto nos dice las diferencias del archivo que se encuentra en el \textit{staging area} y el que tenemos en el \textit{local}. En el caso de queramos añadir un añadir un archivo con un mensaje para los demás desarrolladores ponemos
\begin{verbatim}
$ git commit -m "Mensaje"
\end{verbatim}
Finalmente para subir todo al remoto hacemos
\begin{verbatim}
$ git push
\end{verbatim}
y nos pedire un usuario y una clave.
\begin{figure}[!b]
	\centering
  \includegraphics[scale=0.8]{diagram.jpg}
  \caption{Diagrama básico de GitHub}
  \label{fig:diagram}
\end{figure}

\section*{Ramas}
Existen muchas ramas donde podemos volcar nuestro código sin poner en peligro el general.
Para ello hace falta crear las llamadas \textit{ramas}
\begin{verbatim}
$ git branch <name>
\end{verbatim}
Una vez creadas, podemos cambiar entre ellas simplemente escribiendo
\begin{verbatim}
$ git checkout <name>
\end{verbatim}
Siendo la rama \textbf{\textit{MASTER}}, la más importante de todas. Una buena herramienta para visualizar
los cambios realizados en un archivo es
\begin{verbatim}
$ git diff <name>
\end{verbatim}
Finalmente si estamos en nuestra rama y queremos volcar todos os archivos al master primero
debemos
\begin{verbatim}
$ git merge master
\end{verbatim}
Esta herramienta coge todos los archivos de una rama y los envia a otra, en este caso al estar en nuestra rama
el comando nos dice si existe algun tipo de conflicto entre ramas. Finalmente si queremos enviar todos nuestros archivos de rama al MASTER
Hacemos un cambio  a Master y despues
\begin{verbatim}
$ git merge <name branch>
\end{verbatim}

Cabe destacar la importania de actualizar continuamente los archivos en cada rama
\\begin{verbatim}
$ git pull
\end{verbatim}
%%%%%%%%%%%%%%%%%%%%%%%%%%%%%%%%%%%%%%%%%%%%%%%%%%%%%%%%%%%%%%%%%%%%%%%%%%%%%
%\bibliographystyle{IEEEtran.bst}                                           %
%\bibliography{./bib} % local                                               %
%%%%%%%%%%%%%%%%%%%%%%%%%%%%%%%%%%%%%%%%%%%%%%%%%%%%%%%%%%%%%%%%%%%%%%%%%%%%%
\end{document}
